\section{Exercise 11.47}
\subsection{Prove the successor function for zero to two}
For (successor zero) = one
\begin{align}
							 zero	&= \lambda f. \lambda x. x \\
					successor	&= \lambda n. \lambda f. \lambda x. (f ((n f) x)) \\
		successor(zero)	&= (\lambda n. \lambda f. \lambda x. (f ((n f) x))) (\lambda f. \lambda x. x) \\
										&= (\lambda f. \lambda x. (f ((n f) x))) [n \leftarrow (\lambda f. \lambda x. x)] \\
										&= \lambda f. \lambda x. (f (((\lambda f'. \lambda x'. x') f) x)) \\
										&= \lambda f. \lambda x. (f ((\lambda x'. x') [f' \leftarrow f] x)) \\
										&= \lambda f. \lambda x. (f ((\lambda x'. x') x)) \\
										&= \lambda f. \lambda x. (f (x' [x' \leftarrow x])) \\
										&= \lambda f. \lambda x. (f (x)) \\ 
										&= \lambda f. \lambda x. (f x) \\
										&= one
\end{align}
\emph{Q.E.D.}
\\
For (successor one) = two
\begin{align}
								one	&= \lambda f. \lambda x. (f x) \\
					successor	&= \lambda n. \lambda f. \lambda x. (f ((n f) x)) \\
		 successor(one)	&= (\lambda n. \lambda f. \lambda x. (f ((n f) x))) (\lambda f. \lambda x. (f x)) \\
										&= (\lambda f. \lambda x. (f ((n f) x))) [n \leftarrow (\lambda f. \lambda x. (f x))] \\
										&= \lambda f. \lambda x. (f (((\lambda f'. \lambda x'. (f' x')) f) x)) \\
										&= \lambda f. \lambda x. (f (((\lambda x'. (f' x')) [f' \leftarrow f]) x)) \\
										&= \lambda f. \lambda x. (f ((\lambda x'. (f x')) x)) \\
										&= \lambda f. \lambda x. (f ((f x') [x' \leftarrow x])) \\
										&= \lambda f. \lambda x. (f (f x)) \\
										&= two
\end{align}
\emph{Q.E.D.}

\subsection{Generalization of the successor function}
Let the general lambda abstraction for numbers be
\begin{align}
			church_n = \lambda f. \lambda x. (f^{n} x) \label{e4:church:def}
\end{align}
Such that
\begin{align}
 						f^{0}(x) &= x \\
						f^{1}(x) &= f(x) \\
						f^{2}(x) &= f(f(x)) \\
						f^{3}(x) &= f(f(f(x))) \\
						... \\
					f^{n+1}(x) &= f(f^{n}(x))
\end{align}
For the successor function this requires
\begin{align}
 	successor(church_n) = church_{n+1}
\end{align}
such that
\begin{align}
					 church_n	&= \lambda f. \lambda x. (f^{n} x) \\
successor(church_n)	&= \lambda n. \lambda f. \lambda x. (f ((n f) x)) \\
										&= (\lambda f. \lambda x. (f ((n f) x))) (\lambda f. \lambda x. (f^{n} x)) \\
										&= (\lambda f. \lambda x. (f ((n f) x))) [n \leftarrow (\lambda f. \lambda x. (f^{n} x))] \\
										&= \lambda f. \lambda x. (f (((\lambda f'. \lambda x'. (f'^{n} x')) f) x)) \\
										&= \lambda f. \lambda x. (f (((\lambda x'. (f'^{n} x')) [f' \leftarrow f]) x)) \\
										&= \lambda f. \lambda x. (f ((\lambda x'. (f^{n} x')) x)) \\
										&= \lambda f. \lambda x. (f ((f^{n} x') [x' \leftarrow x])) \\
										&= \lambda f. \lambda x. (f (f^{n} x)) \\
										&= \lambda f. \lambda x. (f^{n+1} x) \\
										&= church_{n+1}
\end{align}
\emph{Q.E.D.}

\subsection{Define addition and multiplication for Church numbers}
The addition for Church numbers can be defined as
\begin{align}
	add	&= \lambda m. \lambda n. \lambda f. \lambda x. (m f) (n f x) \label{eq4:add:proof}
\end{align}
Due to equation \ref{e4:church:def}, it must be that
\begin{align}
						church_m + church_n &= church_{m+n} \\
\Rightarrow \lambda f. \lambda x. (f^{m} x) + \lambda f. \lambda x. (f^{n} x)
																&= \lambda f. \lambda x. (f^{n+m} x) \\
																&= \lambda f. \lambda x. (f^{n}(f^{m} x))
\end{align}
In order for equation \ref{eq4:add:proof} to be true, it must be that
\begin{align}
		add(church_m, church_n) = church_{m+n}
\end{align}
This can be proven as follows:
\begin{align}
							church_m &= \lambda f. \lambda x. (f^{m} x) \\
							church_n &= \lambda f. \lambda x. (f^{n} x) \\
add(church_m,church_n) &= (\lambda m. \lambda n. \lambda f. \lambda x. (m f) (n f x)) (\lambda f. \lambda x. (f^{m} x)) (\lambda f. \lambda x. (f^{n} x)) \\
											 &= (\lambda n. \lambda f. \lambda x. (m f) (n f x)) [m \leftarrow (\lambda f. \lambda x. (f^{m} x))] (\lambda f. \lambda x. (f^{n} x)) \\
											 &= (\lambda n. \lambda f. \lambda x. ((\lambda f'. \lambda x'. (f'^{m} x')) f) (n f x)) (\lambda f. \lambda x. (f^{n} x)) \\
											 &= (\lambda n. \lambda f. \lambda x. ((\lambda x'. (f'^{m} x')) [f' \leftarrow f]) (n f x)) (\lambda f. \lambda x. (f^{n} x)) \\
											 &= (\lambda n. \lambda f. \lambda x. (\lambda x'. (f^{m} x')) (n f x)) (\lambda f. \lambda x. (f^{n} x)) \\
											 &= (\lambda f. \lambda x. (\lambda x'. (f^{m} x')) (n f x)) [n \leftarrow (\lambda f. \lambda x. (f^{n} x)) \\
											 &= \lambda f. \lambda x. (\lambda x'. (f^{m} x')) ((\lambda f''. \lambda x''. (f''^{n} x'')) f x) \\
											 &= \lambda f. \lambda x. (\lambda x'. (f^{m} x')) ((\lambda x''. (f''^{n} x'')) [f'' \leftarrow f] x) \\
											 &= \lambda f. \lambda x. (\lambda x'. (f^{m} x')) ((\lambda x''. (f^{n} x'')) x) \\
											 &= \lambda f. \lambda x. (\lambda x'. (f^{m} x')) ((f^{n} x'') [x'' \leftarrow x]) \\
											 &= \lambda f. \lambda x. (\lambda x'. (f^{m} x')) (f^{n} x) \\
											 &= \lambda f. \lambda x. (f^{m} x') [x' \leftarrow (f^{n} x)] \\
											 &= \lambda f. \lambda x. (f^{m}(f^{n} x)) \\
											 &= \lambda f. \lambda x. (f^{m+n} x) \\
											 &= church_{m+n}
\end{align}
\emph{Q.E.D.}
\\
The multiplication for Church numbers can be easily defined using the $add$ function
\begin{align}
	mult = \lambda m. \lambda n. \lambda f. \lambda x. (m (add (n f x))) x
\end{align}
The proof for this function can be provided analog to the one for the addition. However, since it requires verbose equations, it will be omitted at this point.