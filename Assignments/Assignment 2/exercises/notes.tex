% For all finite $S_A$, it is always that
% \begin{align}
% 						& \exists \text{ finite } X_w \subseteq S_A = S_A \\
% \Rightarrow	&	\sqcup X_w = \sqcup S_A \\
% \Rightarrow & \forall e \leq \sqcup S_A, e \leq \sqcup X_w
% \end{align}
% Hence, Theorem B is true for all finite $S_A$. Also, it is that forall $S_A$ with $\sqcup S_A \in S_A$ it is that
% \begin{align}
% 						& \exists \text{ finite } X_u \subseteq S_A = \sqcup S_A \\
% \Rightarrow	&	\sqcup X_u = \sqcup S_A \\
% \Rightarrow & \forall e \leq \sqcup S_A, e \leq \sqcup X_u \\
% \Rightarrow & \forall e \leq \sqcup S_A, \exists \text{ finite } X \text{ such that } e \leq \sqcup X
% \end{align}
% Thus, Theorem B is true for all $S_A$ with $\sqcup S_A \in S_A$. Next, forall $S_A \in C$, it is that
% \begin{align}
% 						& \forall e \leq \sqcup S_A, 
% 							\exists s \in S_A \text{ such that } e \leq s \\
% \Rightarrow	&	\exists \text{ finite } X_s \subseteq S_A = \{s\} \\
% \Rightarrow & \forall e \leq \sqcup S_A, e \leq \sqcup X_s \\
% \Rightarrow & \forall e \leq \sqcup S_A, \exists \text{ finite } X \text{ such that } e \leq \sqcup X
% \end{align}
% Therefor, Theorem B is true for all $S_A \in C$, meaning all $S_A$ that are a chain. Consequently, only those $S_A$ remain that could disprove Theorem B, for which all of the following apply:
% \begin{enumerate}
% 	\item $S_A$ is not a chain				
% 	\item $\sqcup S_A \notin S_A$			
% 	\item $S_A$ is infinite						
% \end{enumerate}
% In order for a set $S_A$ to have both condition 1 and 2 to be true, it must be that $\exists a,b \in S_A$ such that
% \begin{align}
% 						&	a \nleq b																														\label{eq:cola:theB:abReq:aleqb}	\\
% \text{and }	&	b \nleq a																														\label{eq:cola:theB:abReq:bleqa}	\\
% \text{and }	& \nexists c \in S_A \text{ such that } a \leq c \text{ or } b \leq c	\label{eq:cola:theB:abReq:c}
% \end{align}
% Since if either equation \ref{eq:cola:theB:abReq:aleqb} or \ref{eq:cola:theB:abReq:bleqa} would be false for all $a,b \in S_A$, then $S_A$ would be a chain. If equation \ref{eq:cola:theB:abReq:c} would be false for all $a,b \in S_A$, then it would be that $\sqcup S_A \in S_A$. 



\section{A partially ordered set $D$ is a complete lattice if every subset $S$ of $D$ has a least upper bound $\sqcup S$. Show that in a complete lattice $D$, an element $e$ is compact if and only if for all $S \subseteq D$: $e \leq \sqcup S$ implies $e \leq \sqcup X$ for some finite $X \subseteq S$.}
\subsection{Introduction}
Let $A$ be the set of all subsets of $D$ and $C$ be the set of all subsets of $D$ that are ascending chains such that
\begin{align}
	& A \subseteq D \\
	& C \subseteq D \\
	& C \subseteq A 			\label{eq:cola:def:ca}
\end{align}
It has to be proven that
\begin{align}
	\textbf{iff }		& \forall S_A \in A: \forall e \leq \sqcup S_A,
										\exists \text{ finit } X \subseteq S_A \text{ such that } e \leq \sqcup X \\
	\textbf{then }	& \forall S_C \in C: \forall e \leq \sqcup S_C,
										\exists s \in S_C \text{ such that } e \leq s
\end{align}
for this to be true, two theorems can be deducted that have to be proven:

\subsection{Theorem A}
\begin{align}
	\textbf{if }		& \forall S_A \in A: \forall e \leq \sqcup S_A,
										\exists \text{ finit } X \subseteq S_A \text{ such that } e \leq \sqcup X \\
	\textbf{then }	& \forall S_C \in C: \forall e \leq \sqcup S_C,
										\exists s \in S_C \text{ such that } e \leq s
\end{align}

\subsection{Theorem B}
\begin{align}
	\textbf{if }		& \forall S_C \in C: \forall e \leq \sqcup S_C,
										\exists s \in S_C \text{ such that } e \leq s \\
	\textbf{then }	& \forall S_A \in A: \forall e \leq \sqcup S_A ,
										\exists \text{ finit } X \subseteq S_A \text{ such that } e \leq \sqcup X
\end{align}

\subsection{Proof}
\subsubsection{Proof of Theorem A}
Due to equation \ref{eq:cola:def:ca}, it must be that by satisfying a condition for all $S_A \subseteq A$, it will be satisfied for all $S_C \subseteq C$, too. Consequently it must be that
\begin{align}
	\textbf{if }		& \forall S_A \in A: \forall e \leq \sqcup S_A, 																		\label{eq:cola:pt1:st}
										\exists \text{ finit } X \subseteq S_A \text{ such that } e \leq \sqcup X \\
	\textbf{then }	& \forall S_C \in C: \forall e \leq \sqcup S_C,
										\exists \text{ finit } X \subseteq S_C \text{ such that } e \leq \sqcup X \\			\label{eq:cola:pt1:en}
\end{align}
Since $C$ is a set of ascending chains, it must be that for all finite $X \subseteq S_C$
\begin{align}
						& \exists x \in X = \sqcup X \\
\Rightarrow & \exists x \in S_C = \sqcup X
\end{align}
Therefore
\begin{align}
						&	e \leq \sqcup X \text{ for some finite } X \subseteq S_C \\
\Rightarrow	& e \leq x \text{ for some } x \in \text{ some finite } X \subseteq S_C \\
\Rightarrow & e \leq x \text{ for some } x \in S_C
\end{align}
such that 
\begin{align}
	\textbf{if }		& \forall S_C \in C: \forall e \leq \sqcup S_C,
										exists \text{ finit } X \subseteq S_C \text{ such that } e \leq \sqcup X \\
	\textbf{then }	& \forall S_C \in C: \forall e_C \leq \sqcup S_C,
										\exists s \in S_C \text{ such that } e \leq s 
\end{align}
Due to equation \ref{eq:cola:pt1:st} to \ref{eq:cola:pt1:en}, it must therefore be that
\begin{align}
	\textbf{if }		& \forall S_A \in A: \forall e \leq \sqcup S_A,
										\exists \text{ finit } X \subseteq S_A \text{ such that } e \leq \sqcup X \\
	\textbf{then }	& \forall S_C \in C: \forall e \leq \sqcup S_C,
										\exists s \in S_C \text{ such that } e \leq s
\end{align}
\emph{Q.E.D}. Consequently, Theorem A is true!

\subsubsection{Proof of Theorem B}
For Theorem B to be true it must be that
\begin{align}
	\textbf{if }		& \forall S_C \in C: \forall e \leq \sqcup S_C,																	\label{eq:cola:pt2:st}
										\exists s \in S_C \text{ such that } e \leq s \\
	\textbf{then }	& \forall S_A \in A: \forall e \leq \sqcup S_A,
										\exists \text{ finit } X \subseteq S_A \text{ such that } e \leq \sqcup X			\label{eq:cola:pt2:en}
\end{align}
For every subset $S_A \in A$: Let $Y$ be the set of all $\{s_i,s_j\}$ with $s_i,s_j \in S_A$ such that $ \{s_i,s_j\} \subseteq S_A$. Let furthermore be $Y_i = \bigcup_{\forall y \in Y} y$ and $Y_b = \bigcup_{\forall y \in Y} \{\sqcup y\}$. It must then be that
\begin{align}
						& \sqcup S_A = \sqcup Y_b \\
\Rightarrow & \sqcup S_A = \sqcup Y_i
\end{align}
In case that $Y$ is a finite set, Theorem B must be true since
\begin{align}
						& Y_i \text{ is finite } \\
\Rightarrow	& \exists \text{ finite } X \subseteq S_A = Y_i \text{ such that:} \\
\Rightarrow & \sqcup S_A = \sqcup X \\
\Rightarrow & \forall e \leq \sqcup S_A, e \leq \sqcup X \\
\Rightarrow & \forall e \leq \sqcup S_A, \exists \text{ finite } X \text{ such that } e \leq \sqcup X
\end{align}
and therefore equation \ref{eq:cola:pt2:en} is always true independently from equation \ref{eq:cola:pt2:st}.
In case that $Y$ is infinite, it must be that $(Y_b \cup \sqcup S_A)$ is a directed set since $D$ being a complete lattice results in
\begin{align}
						& \forall d_i,d_j \in D, \exists \sqcup \{d_i,d_j\} \in D \\
\Rightarrow	&	\forall y_i,y_j \in Y_u, \exists \sqcup \{y_i,y_j\} \in (Y_u \cup \sqcup S_A)
\end{align}
Since every directed set always contains at least one ascending chain with the same supremum, it must be that there exists an ascending chain $Y_c \subseteq Y_b$ such that
\begin{align}
						& \sqcup Y_c = \sqcup S_A \\
						& S_A \in C
\end{align}
and due to equation \ref{eq:cola:pt2:st} it must then be that 
\begin{align}
						& \forall e \leq \sqcup S_A, e \leq \sqcup Y_c \\
\Rightarrow & \forall e \leq \sqcup S_A, \exists s \in Y_c \text{ such that } e \leq s \\
\Rightarrow & \forall e \leq \sqcup S_A, \exists s \in Y_b \text{ such that } e \leq s
\end{align}
Since $Y_b = \bigcup_{\forall y \in Y} \sqcup y$, it must be that
\begin{align}
						& \forall s \in Y_b, \exists y \in Y \text{ such that } s \in y \\
\Rightarrow	&	\forall e \leq \sqcup S_A, \exists y \in Y, \text{ such that } \sqcup y \leq e
\end{align}
and since every entry $y \in Y$ is a finite $\{s_i,s_j\} \subseteq S_A$, it must be that
\begin{align}
						& \exists \text{ finite } X \subseteq S_A = y \text{ such that:} \\
\Rightarrow	&	\forall e \leq \sqcup S_A, e \leq X \\
\Rightarrow & \forall e \leq \sqcup S_A, \exists \text{ finite } X \text{ such that } e \leq \sqcup X
\end{align}
Consequently, equation \ref{eq:cola:pt2:en} is true under the condition of equation \ref{eq:cola:pt2:st} for infinite $Y$. Since $S_A$ can only form finite and infinite $Y$, equation \ref{eq:cola:pt2:en} must be true under the condition of equation \ref{eq:cola:pt2:st} for all $S_A$ such that
\begin{align}
	\textbf{if }		& \forall S_C \in C: \forall e \leq \sqcup S_C,
										\exists s \in S_C \text{ such that } e \leq s \\
	\textbf{then }	& \forall S_A \in A: \forall e \leq \sqcup S_A,
										\exists \text{ finit } X \subseteq S_A \text{ such that } e \leq \sqcup X
\end{align}
\emph{Q.E.D.}. Consequently, Theorem B is true!

\subsection{Conclusion}
Since both Theorem A and Theorem B were proven to be true, it original statement from the introduction is correct!