\section{Provide a denotational semantics for ``for'' loops given in Assignment 1.}
% \subsection{old notation}
% \begin{align}
%     & \dsstmt{\forloop{i=E1}{E_2}{E_3}{St}}  \\
% =   & \dsstmt{i=0; \forloop{i=i+E_1}{E_2}{E_3}{St}}   \\
% =   & \dsstmt{\forloop{i=i+E_1}{E_2}{E_3}{St}} \circ \dsstmt{i=0} \\
% =		& \dscond{\dsbool{i < E_2}}
% 						 {\dsstmt{St: \forloop{i=i+E_1}{E_2}{E_3}{St}}}
% 						 {\text{id}} 
% 		  \circ \dsstmt{i=0} \\
% =		& \dscond{\dsbool{i < E_2}}
% 						 {\dsstmt{\forloop{i=i+E_1}{E_2}{E_3}{St}} \circ \dsstmt{St}}
% 						 {\text{id}} 
% 			\circ \dsstmt{i=0}
% \end{align}
% Therefore:
% \begin{align}
% 	  & \dsstmt{\forloop{i=i+E_1}{E_2}{E_3}{St}} \circ \dsstmt{i=0} \\
% =		& \dscond{\dsbool{i < E_2}}
% 						 {\dsstmt{\forloop{i=i+E_1}{E_2}{E_3}{St}} \circ \dsstmt{St}}
% 						 {\text{id}} 
% 			\circ \dsstmt{i=0}
% \end{align}
% Or short:
% \begin{align}
% 	  & \dsstmt{\forloop{i=i+E_1}{E_2}{E_3}{St}} \\
% =		& \dscond{\dsbool{i < E_2}}
% 						 {\dsstmt{\forloop{i=i+E_1}{E_2}{E_3}{St}} \circ \dsstmt{St}}
% 						 {\text{id}} 
% \end{align}
% By expressing this as a fixed point of the functional $F$ we get
% \begin{align}
% 	  & \dsstmt{\forloop{i=i+E_1}{E_2}{E_3}{St}} \\
% =   & \dsfix{F} \\
% 		& \text{where } F g = \dscond{\dsbool{i < E_2}}
% 																 {g \circ \dsstmt{St}}
% 																 {\text{id}} 
% \end{align}
% Since
% \begin{align}
%     & \dsstmt{\forloop{i=E1}{E_2}{E_3}{St}}   \\
% =   & \dsstmt{\forloop{i=i+E_1}{E_2}{E_3}{St}} \circ \dsstmt{i=0}
% \end{align}
% We finally get for the denotational semantic of the \emph{for} loop:
% \begin{align}
%     & \dsstmt{\forloop{i=E1}{E_2}{E_3}{St}}   \\
% =   & \dsfix{F} \circ s[i \to \dsasgn{0}s] \\
% 		& \text{where } F g = \dscond{\dsbool{i < E_2}}
% 																 {g \circ \dsstmt{St}}
% 																 {\text{id}} 
% \end{align}
% 
% \subsection{``attribute grammar''-like notation}
The for loop can for this scenario be rewritten as follows:
\begin{align}
	for	\to & \text{for } i := E_0 \text{ to } E_1 \text{ step } E_2 \text{ do } St \\
			\to & init; loop
\end{align}
where
\begin{align}
	init \to & i := E_0 \\
	inc	 \to & i = i+E_2 \\
	loop \to & \text{while } i <= E_1 \text{ do } (St; inc)
\end{align}
Let S be the set of all statuses. The attribute ``sc'' is the statechange of elements, while ``eval'' evaluates expressions based on a given status. Consequently it must be that
\begin{align}
	for.sc = & \lambda s \in S: (loop.sc \circ init.sc)( s ) \label{eq:for:whole}
\end{align}
with
\begin{align}
	init.sc = & \lambda s \in S: s[i \gets E_0.eval( s )] \\
	inc.sc 	= & \lambda s \in S: s[i \gets s( i ) + E_2.eval( s )] \\
	loop.sc = & \lambda s \in S: \dscond{s( i ) <= E_1.eval( s )}
																			{(loop.sc \circ inc.sc \circ St.sc)( s )}
																			{s} \\
					= & \dsfix{F}\\
where \ F = & \lambda g: \lambda s \in S: \dscond{s( i ) <= E_1.eval( s )}{(g \circ inc.sc \circ St.sc)( s )}{s} \\
					= & \lambda g: \lambda s \in S: \dscond{s( i ) <= E_1.eval( s )}{(g \circ (\lambda x \in S: x[i \gets x( i ) + E_2.eval( x )]) \circ St.sc)(s)}{s}
\end{align}
Putting these into equation \ref{eq:for:whole} results to the denotational semantics for the \emph{for} loop
\begin{align}
	 for.sc = & \lambda s \in S: (\dsfix{F})(s[i \gets E_0.eval( s )]) \\
where \ F = & \lambda g: \lambda s \in S: \dscond{s( i ) <= E_1.eval( s )}{(g \circ (\lambda x \in S: x[i \gets x( i ) + E_2.eval( x )]) \circ St.sc)(s)}{s}
\end{align}